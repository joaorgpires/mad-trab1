
\documentclass[paper=a4, fontsize=11pt]{scrartcl}
\usepackage[T1]{fontenc}
\usepackage{fourier}

\usepackage[english]{babel}															% English language/hyphenation
\usepackage[protrusion=true,expansion=true]{microtype}	
\usepackage{amsmath,amsfonts,amsthm} % Math packages
\usepackage[pdftex]{graphicx}	
\usepackage{url}
\usepackage{verbatim}
\usepackage{mathtools}


%%% Custom sectioning
\usepackage{sectsty}
\allsectionsfont{\centering \normalfont\scshape}


%%% Custom headers/footers (fancyhdr package)
\usepackage{fancyhdr}
\pagestyle{fancyplain}
\fancyhead{}											% No page header
\fancyfoot[L]{}											% Empty 
\fancyfoot[C]{}											% Empty
\fancyfoot[R]{\thepage}									% Pagenumbering
\renewcommand{\headrulewidth}{0pt}			% Remove header underlines
\renewcommand{\footrulewidth}{0pt}				% Remove footer underlines
\setlength{\headheight}{13.6pt}


%%% Equation and float numbering
\numberwithin{equation}{section}		% Equationnumbering: section.eq#
\numberwithin{figure}{section}			% Figurenumbering: section.fig#
\numberwithin{table}{section}				% Tablenumbering: section.tab#


%%% Maketitle metadata
\newcommand{\horrule}[1]{\rule{\linewidth}{#1}} 	% Horizontal rule

\title{
		\usefont{OT1}{bch}{b}{n}
		\normalfont \normalsize \textsc{Faculdade de Ci�ncias da Universidade do Porto} \\ [25pt]
		\horrule{0.5pt} \\[0.4cm]
		\Huge M�todos de Apoio � Decis�o - Trabalho 1 \\
		\horrule{2pt} \\[0.5cm]
}
\author{
\\*
		\normalfont 								
\large
        Jo�o Pedro Pereira dos Santos 201305900\\[-3pt]		\large
        Jo�o Rebelo Grifo Pires 201200384\\ [-3pt]
\large
		\today 
	   }
\date{}


%%% Begin document
\begin{document}


\maketitle

\newpage
\section{\textbf{Mostre que sem a restri��o de capacidade, e com
esta estrutura de custos, o problema de minimizar
os custos de transporte � trivial.}}

\qquad	Considerando que n�o temos nenhum tipo de restri��o de capacidade, a solu��o � sempre trivial pois a dist�ncia m�nima entre dois planetas (que reflete tamb�m o custo m�nimo) � encontrada tra�ando uma trajet�ria reta e direta  desde o planeta em que nos encontramos para o planeta para que queremos ir, levando apenas os recursos que tencionamos trocar.

\section{\textbf{Formule em programa��o matem�tica o problema
anterior (i.e., sem restri��es de capacidade).
Resolva-o, e mostre que a solu��o � a esperada.}}

\textit{Vari�veis de Decis�o:}
\\*
\qquad 
- \textbf{X}i,j: Quantidade de unidades que vamos enviar entre um par de planetas, planeta de �ndice i e planeta de �ndice j; \\* 
- \textbf{C}i,j: Custo para movimentar entre um par de planetas, planeta de �ndice i e planeta de �ndice j; \\* 
- \textbf{x}: Coordenada cartesiana referente � disposi��o de cada um dos planetas no eixo dos xx; \\* 
- \textbf{y}: Coordenada cartesiana referente � disposi��o de cada um dos planetas no eixo dos yy; \\* \\*

\textit{Restri��es: 
\\* - \textbf{X}i,j >= 0; 
\\* - x, y >= 0;
\\* -$\displaystyle \sum_{i,j=1}^{25} \textbf{X}j,i - \sum_{i,j=1}^{25} \textbf{X}i,j$ = \textbf{d}i}
\\* \\*

\textit{Fun��o Objectivo:} \\*
\qquad \textbf{min z} = $\displaystyle \sum_{i,j=1}^{25} \textbf{C}i,j * \textbf{X}i,j$

\end{document}